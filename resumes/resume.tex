%%% Local Variables:
%%% coding: utf-8
%%% mode: latex
%%% TeX-engine: xetex
%%% End:
\documentclass{tccv}
\usepackage[english]{babel}
\usepackage[none]{hyphenat}
\renewcommand{\bf}{\textbf}
\renewcommand{\sc}{\textsc}
\renewcommand{\it}{\textit}

\begin{document}

\part{Shotaro Ikeda}

\personal
[https://shotaroikeda.github.io/]
{APT 5 404 E. Stoughton Ave \newline
  Champaign, IL 61820}
{+1 (408) 513-5376}
{ikeda2@illinois.edu}

\section{Work experience}

\begin{eventlist}

  \eventData{June 2015 -- Present}
  {CS 196}
  {Course Assistant}

  \begin{itemize}
  \item Currently writing homework assignments for students, very active helping students on Piazza.
  \item Managed two projects, \href{https://github.com/SNAPPETITE}{Snappettite} and \href{https://github.com/InterestMatcher}{Interest Matcher}.
  \item Lead Artificial Intelligence Hackerspace, taught Freshman how to use the Naive Bayes Classifier to process and use the MNIST dataset.
  \end{itemize}

  \eventData{August 2015 -- Present}
  {HackIllinois}
  {Mobile/Backend Developer}
  \begin{itemize}
  \item Engaged in the ``Open Hackathon'' initiative.
  \item Currently lead developer of the official iOS Application and contributing to \\
    backend development.
  \item Administered official cluehunt application in 2015. iOS version had 51 users.
  \end{itemize}

\end{eventlist}

\section{Education}

\begin{yearlist}

  \education{B.S. Computer Science}{2014 - Present}
  {University of Illinois at  \newline Urbana-Champaign}
  {GPA: 3.63 / 4.0}
  {Graduation: May 2018}

\end{yearlist}

\section{Relevant Coursework}
\begin{courselist}

  \eventData{Courses Taken}
  \begin{tabular}{l l}
    CS 241 & Systems Programming \\
    CS 421 & Programming Languages \\
  \end{tabular}

  \eventData{Current Courses}
  \begin{tabular}{l l}
    CS 374 & Algos. and Models of Computation \\
    CS 427 & Software Engineering I \\
    CS 461 & Computer Security I \\
    CS 498SL3 & Virtual Reality \\
  \end{tabular}
  \forceskip
  Full list available on \href{https://shotaroikeda.github.io/}{my website}.

\end{courselist}

\section{projects}

\begin{projectlist}

  \project{May 2016 -- Present}
  {HackIllinois iOS App}

  \begin{itemize}
  \item Current project for HackIllinois. Open Source.
  \item Features basic event features for Hackathons.
  \end{itemize}

  \project{May 2016 -- Present}
  {MoodTrackr}
  \begin{itemize}
  \item W.I.P. Allows you to see what kind of moods are around using sentiment analysis via decision tree.
  \item Data processing is currently done, using Python's multiprocessing library (to circumvent GIL).
  \end{itemize}

  \project{June 2016}
  {LiquidActionButton}
  \begin{itemize}
  \item Open source project. An iOS UIButton-like class inspired by material design.
  \item Added more versatility and obtained small \\
    performance gain, about 5FPS.
  \end{itemize}

  \project{October 2015}
  {HandReader2}
  \begin{itemize}
  \item Created as a tutorial for students in CS 196.
  \item A revisit of HandReader, using newfound Numpy knowledge. About 10 seconds faster than the original.
  \item 84.3\% accuracy using the MNIST Database.
  \end{itemize}

  \project{Feburary 2015}
  {Flash Me!}
  \begin{itemize}
  \item SpartaHack 2016 Submission.
  \item Created iOS application, created weighting \\
    algorithm to increase the probability of showing cards that were marked incorrect.
  \end{itemize}

  \project{September 2016}
  {Regex Cross-Compiler}
  \begin{itemize}
  \item Fun side project to cross compile Mathmatical Regular Expressions to Python Regex.
  \item Generates syntax tree to parse and \\
    transform into Python Regex.
  \item Wishful TODO: auto-optimization of regex.
  \end{itemize}

\end{projectlist}

\section{Languages}

\begin{factlist}

  \eventData{Comfortable}
  {C, Swift, and Python}

  \eventData{Previously Used}
  {JavaScript, CSS, HTML, Clojure, Haskell, and LaTeX}

  \eventData{Used in Classes}
  {Java and C++}

\end{factlist}

\section{Interests}
\begin{itemize}
\item Machine Learning, Artificial Intelligence, \\
  Backend, and Full-stack.
\item Creative work, difficult, non-trival, or \\
  challenging problems.
\end{itemize}

\end{document}
