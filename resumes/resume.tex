%%% Local Variables:
%%% coding: utf-8
%%% mode: latex
%%% TeX-engine: xetex
%%% End:
\documentclass[9pt,a4paper,sans]{moderncv}


% Modern CV Style
\moderncvstyle{classic}
\moderncvcolor{purple}

\usepackage{color}

\usepackage[english]{babel}
\usepackage[none]{hyphenat}

% Color definitions
\definecolor{kwcolor}{RGB}{229,81,107}
\definecolor{hrefcolor}{RGB}{90,90,232}

%% Change main color to maroon
\definecolor{color0}{RGB}{0,0,0}
\definecolor{color1}{RGB}{136,22,22}

\newcommand{\kw}[1]{%
  #1%
}%

\newcommand{\chref}[2]{%
  \textit{\href{#1}{#2}}%
}%

\newenvironment{itemize*}%
{\begin{itemize}%
    \vspace{-0.7em}
    \setlength{\itemsep}{0pt}%
    \setlength{\parskip}{0pt}}%
  {\end{itemize}}

\usepackage[utf8]{inputenc}                   % replace by the encoding you are using

% adjust the page margins
\usepackage[scale=0.9]{geometry}

\firstname{Shotaro}
\familyname{Ikeda}
\email{ikeda2@illinois.edu}                      % optional, remove the line if not wanted
\homepage{shotaroikeda.github.io}                % optional, remove the line if not wanted
\social[linkedin]{shotaroikeda}                        % optional, remove / comment the line if not wanted
\quote{\small Studious, Ambitious, and Innovative Artificial Intelligence Researcher.}
% \extrainfo{\url{http://markliu.me}} % optional, remove the line if not wanted
\begin{document}
\maketitle

\section{Education}
\cventry{2014--Now}{BS. Computer Science}{University of Illinois at Urbana-Champaign}{Urbana, IL}{}{}
\cvline{GPA}{\small 3.72/4.0}
\cvline{Graduation}{May 2018}

\section{Work experience}
\cventry{2017--Now}{Undergraduate Research Assistant}{University of Illinois at Urbana-Champaign}{Urbana, IL}{}{%
  \begin{itemize*}
  \item Research to give agents spatial awareness.
  \item Speed up extra supervision networks for \kw{Reinforcement Learning}.
  \end{itemize*}
}%

\cventry{Summer 2017}{Technical Development Program Intern}{Capital One}{Richmond, VA}{}{%
  \begin{itemize*}
  \item Fast, repeatable, and robust solution for comparing data quality of multiple competing services.
  \item Database comparisons using \kw{Python}, \kw{Numpy}, \kw{Pandas}, and \kw{Matplotlib}.
  \item Engaged proof of concept for logo recognition using \kw{Deep Learning} with \kw{Keras}.
  \item Provided mentor-ship to increase validation accuracy from 85\% to 96\%.
  \end{itemize*}
}%

\cventry{2015--2017}{Course Assistant}{CS 196}{Champaign, IL}{}{%
  \begin{itemize*}
  \item Involved in making course materials, quality assurance, and acted as one of the instructors of the class.
  \item Gave lectures in \kw{Theory}, \kw{Algorithms}, \kw{Recursion}, \kw{AI}, and \kw{Machine Learning}.
  \end{itemize*}
}%

\cventry{Winter 2016}{Intern}{Double Sharp Plus Co. Ltd}{Hachiouji, Japan}{}{%
  \begin{itemize*}
  \item \kw{Object Character Recognition} on number plates, mined my own dataset
  \item Histogram approach had 60\% accuracy on test set I provided
  \item \kw{Support Vector Machine} Classifier using Local Binary Pattern algorithm, with ~50\% accuracy
  \item Implemented \kw{Deep Convolutional Neural Network} in \kw{Tensorflow} based off of CAPTCHA, with ~80\% accuracy
  \end{itemize*}
}%

\section{Projects}
\cventry{2017--Now}{DFP-PyTorch}{Undergrad Research Assistant}{Uni. of Illinois at Urbana-Champaign}{Urbana, IL}{%
  \begin{itemize*}
  \item State of the art agent that won \href{http://vizdoom.cs.put.edu.pl/competition-cig-2016/results}{ViZDoom AI Competition 2016 (Deathmatch)}.
  \item Implemented \chref{https://arxiv.org/abs/1611.01779}{Learning to Act by Predicting the Future} (Dosovitskiy, et. al) in \kw{PyTorch}.
  \item Implemented general experimentation architecture for future research.
  \item Implemented a generalization for multiple sparse input streams.
  \end{itemize*}
}

\cventry{Summer 2017}{FCRN-PyTorch}{Undergrad Research Assistant}{Uni. of Illinois at Urbana-Champaign}{Urbana, IL}{%
  \begin{itemize*}
  \item State of the art Depth Prediction Network.
  \item Implemented \chref{https://arxiv.org/abs/1606.00373}{Deeper Depth Prediction with Fully Convolutional Residual Networks} (Laina, et. al) in \kw{PyTorch}.
  \item Provided additional base layers (ResNet-18, ResNet-34) for further experimentation.
  \end{itemize*}
}

\cventry{Spring 2017}{CartPole-DQN}{Personal}{}{}{ %
  \begin{itemize*}
  \item Implemented a \kw{Deep Q-Network} in \kw{PyTorch}.
  \item Experimented with Reward Shaping to provide additional supervision and help convergence.
  \end{itemize*}
}


\cventry{Spring 2017}{Wide Residual Network}{Personal}{}{}{%
  \begin{itemize*}
  \item A variation of the cutting-edge ResNet architecture.
  \item Implemented \chref{https://arxiv.org/abs/1605.07146}{Wide Residual Networks} (Zagoruyko, et. al) in \kw{Tensorflow}.
  \item Obtained ~93.9\% accuracy training on CIFAR-10.
  \end{itemize*}
}

\cventry{Spring 2017}{Autoencoders}{Personal}{}{}{%
  \begin{itemize*}
  \item Implemented a \kw{Variational Autoencoder} (VAE) and \kw{Stacked Denoising Autoencoder} in \kw{Tensorflow}.
  \item Experimented with different activation functions and latent variable sizes.
  \end{itemize*}
}
\end{document}

%% Feedback
% Numbers are really good, add more if possible
% Mention paper versatility in elevator pitch/tagline
% Differentiate between types of project (personal/research, etc)