%%% Local Variables:
%%% coding: utf-8
%%% mode: latex
%%% TeX-engine: xetex
%%% End:
\documentclass{tccv}
\usepackage[english]{babel}
\usepackage[none]{hyphenat}
\renewcommand{\bf}{\textbf}
\renewcommand{\sc}{\textsc}
\renewcommand{\it}{\textit}

\newenvironment{itemize*}%
{\begin{itemize}%
    \vspace{-0.7em}
    \setlength{\itemsep}{0pt}%
    \setlength{\parskip}{0pt}}%
  {\end{itemize}}
\begin{document}

\part{Shotaro Ikeda}

\personal
[shotaroikeda.github.io]
{APT 5 404 E. Stoughton Ave \newline
  Champaign, IL 61820}
{+1 (408) 513-5376}
{ikeda2@illinois.edu}

\section{Education}

\begin{yearlist}

  \education{B.S. Computer Science}{2014 - Present}
  {University of Illinois at  \newline Urbana-Champaign}
  {GPA: 3.63 / 4.0}
  {Graduation: May 2018}

\end{yearlist}

\section{Work experience}

\begin{eventlist}

  \eventData{June 2015 -- Present}
  {CS 196}
  {Course Assistant}

  \begin{itemize*}
  \item Currently writing homework assignments for students, very active helping students.
  \item Managed three projects, \href{https://github.com/SNAPPETITE}{Snappettite}, \href{https://github.com/InterestMatcher}{Interest Matcher}, and currently SentiMiner.
  \item Lead Artificial Intelligence Hackerspace, taught Freshman how to use the Naive Bayes Classifier to process and use the MNIST dataset.
  \end{itemize*}

  \eventData{August 2015 -- Present}
  {HackIllinois}
  {Mobile/Backend Developer}
  \begin{itemize*}
  \item Engaged in the ``Open Hackathon'' initiative.
  \item Currently lead developer of the official iOS Application and contributing to \\
    backend development.
  \item Administered official cluehunt application in 2015. iOS version had 51 users.
  \end{itemize*}

\end{eventlist}

\section{Relevant Coursework}
\begin{courselist}

  \eventData{Courses Taken}
  \begin{tabular}{l l}
    CS 241 & Systems Programming \\
    CS 421 & Programming Languages \\
  \end{tabular}

  \eventData{Current Courses}
  \begin{tabular}{l l}
    CS 374 & Algos. and Models of Computation \\
    CS 427 & Software Engineering I \\
    CS 461 & Computer Security I \\
    CS 498SL3 & Virtual Reality \\
  \end{tabular}
  \forceskip
  Full list available on \href{https://shotaroikeda.github.io/}{my website}.

\end{courselist}

\section{projects}

\begin{projectlist}

  \project{October 2016 -- November 2016}
  {HandReader3}
  \begin{itemize*}
  \item Achieved ~95\% with regular netural network (784 Input nodes, 10 hidden nodes, 10 output nodes).
  \item 99.2\% using Softmax and Convolutions, using Tensorflow.
  \end{itemize*}

  \project{November 2016 -- Present}
  {Hive}
  \begin{itemize*}
  \item Wildhacks 2016 Entry, predict final Reddit scores given post title
  \item Exploration of Tensorflow+NLP was the main goal, figuring out why Bag-of-Words is considered outdated.
  \item Bag of Words and 1-gram model, input vector size was 10009.
  \item Some posts were predicted well (within 400 votes) while others were off (some were predicted negative).
  \end{itemize*}

  \project{November 2016 -- Present}
  {Titanic}
  \begin{itemize*}
  \item Kaggle Competition Entry, predicting who survives on the Titanic
  \item 76\% accuracy using Random Forest Classifier, no hypertuning
  \end{itemize*}

  \project{May 2016 -- Present}
  {HackIllinois iOS App}
  \begin{itemize*}
  \item Current project for HackIllinois. Open Source.
  \item Features basic event features for Hackathons.
  \end{itemize*}

  \project{September 2016}
  {Regex Cross-Compiler}
  \begin{itemize*}
  \item Fun side project to cross compile Mathmatical Regular Expressions to Python Regex.
  \item Generates syntax tree to parse and \\
    transform into Python Regex.

  \end{itemize*}
  \project{June 2016}
  {LiquidActionButton}
  \begin{itemize*}
  \item Open source project. Material design button ported to iOS.
  \item Added more versatility and obtained small \\
    performance gain, about 5FPS.
  \end{itemize*}

  \project{Feburary 2015}
  {Flash Me!}
  \begin{itemize*}
  \item SpartaHack 2016 Submission.
  \item Created iOS application, created weighting \\
    algorithm to increase the probability of showing cards that were marked incorrect.
  \end{itemize*}

\end{projectlist}

\section{Languages}

\begin{factlist}

  \eventData{Comfortable}
  {C, Swift, and Python}

  \eventData{Previously Used}
  {JavaScript, CSS, HTML, Clojure, Haskell, and LaTeX}

  \eventData{Used in Classes}
  {Java and C++}

\end{factlist}

\section{Interests}
\begin{itemize*}
\item Machine Learning, Artificial Intelligence, \\
  Backend, and Full-stack.
\item Creative work, difficult, non-trival, or \\
  challenging problems.
\end{itemize*}

\end{document}
